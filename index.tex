% Options for packages loaded elsewhere
\PassOptionsToPackage{unicode}{hyperref}
\PassOptionsToPackage{hyphens}{url}
\PassOptionsToPackage{dvipsnames,svgnames,x11names}{xcolor}
%
\documentclass[
  12pt,
  letterpaper,
  DIV=11,
  numbers=noendperiod]{scrartcl}

\usepackage{amsmath,amssymb}
\usepackage{setspace}
\usepackage{iftex}
\ifPDFTeX
  \usepackage[T1]{fontenc}
  \usepackage[utf8]{inputenc}
  \usepackage{textcomp} % provide euro and other symbols
\else % if luatex or xetex
  \usepackage{unicode-math}
  \defaultfontfeatures{Scale=MatchLowercase}
  \defaultfontfeatures[\rmfamily]{Ligatures=TeX,Scale=1}
\fi
\usepackage{lmodern}
\ifPDFTeX\else  
    % xetex/luatex font selection
  \setsansfont[]{Times New Roman}
\fi
% Use upquote if available, for straight quotes in verbatim environments
\IfFileExists{upquote.sty}{\usepackage{upquote}}{}
\IfFileExists{microtype.sty}{% use microtype if available
  \usepackage[]{microtype}
  \UseMicrotypeSet[protrusion]{basicmath} % disable protrusion for tt fonts
}{}
\makeatletter
\@ifundefined{KOMAClassName}{% if non-KOMA class
  \IfFileExists{parskip.sty}{%
    \usepackage{parskip}
  }{% else
    \setlength{\parindent}{0pt}
    \setlength{\parskip}{6pt plus 2pt minus 1pt}}
}{% if KOMA class
  \KOMAoptions{parskip=half}}
\makeatother
\usepackage{xcolor}
\usepackage[left=3cm, top=3cm, right=2cm, bottom=2cm]{geometry}
\setlength{\emergencystretch}{3em} % prevent overfull lines
\setcounter{secnumdepth}{-\maxdimen} % remove section numbering
% Make \paragraph and \subparagraph free-standing
\ifx\paragraph\undefined\else
  \let\oldparagraph\paragraph
  \renewcommand{\paragraph}[1]{\oldparagraph{#1}\mbox{}}
\fi
\ifx\subparagraph\undefined\else
  \let\oldsubparagraph\subparagraph
  \renewcommand{\subparagraph}[1]{\oldsubparagraph{#1}\mbox{}}
\fi


\providecommand{\tightlist}{%
  \setlength{\itemsep}{0pt}\setlength{\parskip}{0pt}}\usepackage{longtable,booktabs,array}
\usepackage{calc} % for calculating minipage widths
% Correct order of tables after \paragraph or \subparagraph
\usepackage{etoolbox}
\makeatletter
\patchcmd\longtable{\par}{\if@noskipsec\mbox{}\fi\par}{}{}
\makeatother
% Allow footnotes in longtable head/foot
\IfFileExists{footnotehyper.sty}{\usepackage{footnotehyper}}{\usepackage{footnote}}
\makesavenoteenv{longtable}
\usepackage{graphicx}
\makeatletter
\def\maxwidth{\ifdim\Gin@nat@width>\linewidth\linewidth\else\Gin@nat@width\fi}
\def\maxheight{\ifdim\Gin@nat@height>\textheight\textheight\else\Gin@nat@height\fi}
\makeatother
% Scale images if necessary, so that they will not overflow the page
% margins by default, and it is still possible to overwrite the defaults
% using explicit options in \includegraphics[width, height, ...]{}
\setkeys{Gin}{width=\maxwidth,height=\maxheight,keepaspectratio}
% Set default figure placement to htbp
\makeatletter
\def\fps@figure{htbp}
\makeatother
% definitions for citeproc citations
\NewDocumentCommand\citeproctext{}{}
\NewDocumentCommand\citeproc{mm}{%
  \begingroup\def\citeproctext{#2}\cite{#1}\endgroup}
\makeatletter
 % allow citations to break across lines
 \let\@cite@ofmt\@firstofone
 % avoid brackets around text for \cite:
 \def\@biblabel#1{}
 \def\@cite#1#2{{#1\if@tempswa , #2\fi}}
\makeatother
\newlength{\cslhangindent}
\setlength{\cslhangindent}{1.5em}
\newlength{\csllabelwidth}
\setlength{\csllabelwidth}{3em}
\newenvironment{CSLReferences}[2] % #1 hanging-indent, #2 entry-spacing
 {\begin{list}{}{%
  \setlength{\itemindent}{0pt}
  \setlength{\leftmargin}{0pt}
  \setlength{\parsep}{0pt}
  % turn on hanging indent if param 1 is 1
  \ifodd #1
   \setlength{\leftmargin}{\cslhangindent}
   \setlength{\itemindent}{-1\cslhangindent}
  \fi
  % set entry spacing
  \setlength{\itemsep}{#2\baselineskip}}}
 {\end{list}}
\usepackage{calc}
\newcommand{\CSLBlock}[1]{\hfill\break\parbox[t]{\linewidth}{\strut\ignorespaces#1\strut}}
\newcommand{\CSLLeftMargin}[1]{\parbox[t]{\csllabelwidth}{\strut#1\strut}}
\newcommand{\CSLRightInline}[1]{\parbox[t]{\linewidth - \csllabelwidth}{\strut#1\strut}}
\newcommand{\CSLIndent}[1]{\hspace{\cslhangindent}#1}

\usepackage{fancyhdr}
\usepackage{anyfontsize}
\pagestyle{fancy}
\fancyhf{}
\renewcommand{\headrulewidth}{0pt} 
\fancyfoot[R]{\thepage} 
\KOMAoption{captions}{tableheading}
\makeatletter
\@ifpackageloaded{caption}{}{\usepackage{caption}}
\AtBeginDocument{%
\ifdefined\contentsname
  \renewcommand*\contentsname{Table of contents}
\else
  \newcommand\contentsname{Table of contents}
\fi
\ifdefined\listfigurename
  \renewcommand*\listfigurename{List of Figures}
\else
  \newcommand\listfigurename{List of Figures}
\fi
\ifdefined\listtablename
  \renewcommand*\listtablename{List of Tables}
\else
  \newcommand\listtablename{List of Tables}
\fi
\ifdefined\figurename
  \renewcommand*\figurename{Figura}
\else
  \newcommand\figurename{Figura}
\fi
\ifdefined\tablename
  \renewcommand*\tablename{Tabela}
\else
  \newcommand\tablename{Tabela}
\fi
}
\@ifpackageloaded{float}{}{\usepackage{float}}
\floatstyle{ruled}
\@ifundefined{c@chapter}{\newfloat{codelisting}{h}{lop}}{\newfloat{codelisting}{h}{lop}[chapter]}
\floatname{codelisting}{Listing}
\newcommand*\listoflistings{\listof{codelisting}{List of Listings}}
\makeatother
\makeatletter
\makeatother
\makeatletter
\@ifpackageloaded{caption}{}{\usepackage{caption}}
\@ifpackageloaded{subcaption}{}{\usepackage{subcaption}}
\makeatother
\ifLuaTeX
  \usepackage{selnolig}  % disable illegal ligatures
\fi
\usepackage{bookmark}

\IfFileExists{xurl.sty}{\usepackage{xurl}}{} % add URL line breaks if available
\urlstyle{same} % disable monospaced font for URLs
\hypersetup{
  pdftitle={Investigando Fatores Críticos no Desempenho de Fundos de Investimento Imobiliário},
  colorlinks=true,
  linkcolor={blue},
  filecolor={Maroon},
  citecolor={Blue},
  urlcolor={Blue},
  pdfcreator={LaTeX via pandoc}}

\title{\fontsize{12pt}{14pt}\selectfont Investigando Fatores Críticos no
Desempenho de Fundos de Investimento Imobiliário}
\author{\fontsize{12pt}{14pt}\selectfont Marcus Antonio Cardoso
Ramalho \and \fontsize{12pt}{14pt}\selectfont Ariel Levy}
\date{}

\begin{document}
\maketitle

\setstretch{1.5}
\thispagestyle{fancy}

\hypersetup{
    colorlinks=true,
    linkcolor=black,
    filecolor=black,      
    urlcolor=black,
    citecolor=black,
}

\subsection{Introdução}\label{introduuxe7uxe3o}

O mercado de fundos de investimento imobiliário no Brasil nos últimos
anos tem apresentado um crescimento notável, tanto em relação ao número
de novos fundos listados quanto em relação ao número de investidores.
Isso fica evidente ao se analisar os números presentes nos relatórios
mensais da B3, que denotam uma tendência positiva principalmente a
partir de 2016 (B3, 2023). Apesar de ter um nome diferente no Brasil os
FIIs possuem características similares aos REITs (Real Estate Investment
Trusts) norte-americanos, que são fundos de investimento imobiliário
negociados em bolsa de valores, além disso, este tipo de instrumento
financeiro é bastante popular em diversos países ao redor do mundo desde
de sua criação em 1960 nos Estados Unidos através do REIT Act, assinado
pelo presidente Dwight D. Eisenhower (Block, 2012).

Os FIIs foram criados em 1993 no Brasil e são regulamentados pela
Comissão de Valores Mobiliários (CVM) que é o órgão responsável por
regular e fiscalizar o mercado de capitais brasileiro. Assim como os
REITs, os FIIs são fundos que em ativos imobiliários e que tem por
característica principal a distribuição da maior parte de seus lucros
aos cotistas e a isenção de imposto de renda sobre os rendimentos
distribuídos.

Acompanhando o crescimento do mercado de fundos imobiliários,
pesquisadores tem se interessado cada vez mais em investigar este tipo
de ativo. Conforme apontado por Weise et al. (2017), o número de
publicações sobre o tema apresentou um crescimento constante a partir do
início dos anos 2000 o que vai de encontro aos apontamentos de Block
(2012) que comenta sobre como os REITs se consolidaram como um
instrumento financeiro popular e de grande interesse no mesmo período.

Assim como qualquer tipo de ativo financeiro, os fundos imobiliários
estão sujeitos a riscos inerentes tanto ao próprio mercado quanto a
fatores macroeconômicos que variam de acordo com o país. Em uma análise
bibliométrica sobre o tema, Teófilo et al. (2022-10-10, 2022-10)
identificou indícios de de uma certa concentração sobre temas de artigos
relacionados ao sentimento dos investidores e a análise de desempenho
dos fundos quanto ao risco e retorno. Visto que esse tipo de análise já
tem sido explorada amplamente na literatura, abre-se a oportunidade para
investigar outros fatores que possam influenciar o desempenho dos fundos
imobiliários, como por exemplo a influência de fatores macroeconômicos e
influência de outras variáveis que geralmente não são consideradas em
estudos sobre o tema como a popularidade dos fundos entre os
investidores.

Considerando a importância do tema e a popularidade dos fundos
imobiliários no Brasil e no mundo este trabalho tem como objetivo
explorar as variáveis que influenciam os fundos de investimento
imobiliário quanto ao seu desempenho e a sua relação com o crescimento
do mercado ao longo do tempo com foco no efeito das variáveis
macroeconômicas, popularidade e pagamento de dividendos como fatores
críticos no desempenho ou retorno dos fundos.

Para isso, este trabalho se propõe a desenvolver uma revisão sitemática
da literatura usando o método PRISMA (Page et al., 2021-03-29, 2021-03)
nas bases indexadas Web of Science, Scopus e Periódicos Capes.

Para alcançar os objetivos propostos, foram consideradas as seguintes
perguntas de pesquisa: Como a distribuição de dividendos afeta os
retornos? Como os fundos reagem a períodos de crise econômica? Como
mudanças na economia afetam o desempenho? Como a popularidade é afetada
pelo Youtube? Quais são as variáveis mais estudadas na literatura?

\subsection{Metodologia}\label{metodologia}

Este trabalho utiliza os princípios do método PRISMA, este método

\subsection*{Referências}\label{referuxeancias}
\addcontentsline{toc}{subsection}{Referências}

\phantomsection\label{refs}
\begin{CSLReferences}{1}{0}
\bibitem[\citeproctext]{ref-b32023}
B3. (2023). \emph{Boletim mensal {FII}}.
\url{https://www.b3.com.br/data/files/59/20/79/4A/2E73E810D34843E8AC094EA8/Boletim\%20FII\%20-\%2002M24.pdf}

\bibitem[\citeproctext]{ref-ahistor2012}
Block, R. L. (2012). A history of REITs and REIT performance.
\emph{Investing in REITs}, 109--140.
\url{https://doi.org/10.1002/9781119202325.ch6}

\bibitem[\citeproctext]{ref-page2021}
Page, M. J., Moher, D., Bossuyt, P. M., Boutron, I., Hoffmann, T. C.,
Mulrow, C. D., Shamseer, L., Tetzlaff, J. M., Akl, E. A., Brennan, S.
E., Chou, R., Glanville, J., Grimshaw, J. M., Hróbjartsson, A., Lalu, M.
M., Li, T., Loder, E. W., Mayo-Wilson, E., McDonald, S., \ldots{}
McKenzie, J. E. (2021-03-29, 2021-03). {PRISMA} 2020 explanation and
elaboration: Updated guidance and exemplars for reporting systematic
reviews. \emph{BMJ (Clinical Research Ed.)}, n160.
\url{https://doi.org/10.1136/bmj.n160}

\bibitem[\citeproctext]{ref-teuxf3filo2022}
Teófilo, P. L. B. D. C., Souza, H. H. D., Silva, L. P. C. D., \&
Bergiante, N. C. R. (2022-10-10, 2022-102022-10-10, 2022-10).
\emph{{ENEGEP 2022 - Encontro Nacional de Engenharia de Produ{ç}{ã}o}}.
\url{https://doi.org/10.14488/ENEGEP2022_TN_ST_384_1902_45136}

\bibitem[\citeproctext]{ref-weise2017}
Weise, A. D., De Freitas Battisti, J., Minosso, A. M., Minosso, F., \&
Burgin, J. (2017). \emph{{17ª Confer{ê}ncia Internacional da LARES}}.
\url{https://doi.org/10.15396/lares_2017_paper_17}

\end{CSLReferences}



\end{document}
